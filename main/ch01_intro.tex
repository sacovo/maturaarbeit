\chapter{Vorwort}
\section{Test}
Using Sage\TeX, one can use Sage to compute things and put them into
your \LaTeX{} document. For example, there prime factors of $1269$ ar $\sage{factor(1269)}$.
You don't need to compute the number yourself, or even cut and paste
it from somewhere.

Here's some Sage code:

\begin{sageblock}
    f(x) = exp(x) * sin(2*x)
\end{sageblock}

The second derivative of $f$ is

\[
  \frac{\mathrm{d}^{2}}{\mathrm{d}x^{2}} \sage{f(x)} =
  \sage{diff(f, x, 2)(x)}.
\]

Figure~\ref{fig:sage} is a plot of $f$ from $-1$ to $1$:
\begin{figure}[h]
\caption{A plot of $f$ from $-1$ to $1$.}
\centering
\label{fig:sage}
\sageplot[width=.75\textwidth]{plot(f, -1, 1)}
\end{figure}

\begin{figure}
\caption{A plot of $sin(x)$}
\centering
\label{fig:sin}
\sageplot[width=.75\textwidth]{plot(sin(x), x, 0, 2*pi)}
\end{figure}

\section{Glossary-Test}
\Gls{naiive} people don't know about
alternative \gls{computer} operating systems:
\glspl{Linux}, BSDs and GNU/Hurd.
